\begin{problem}
Есть $N$ чуваков, какие-то из них честные, какие-то нет. Известно, что среди них больше $\frac{N}{2}$ честных. Все $N$ чуваков знают друг о друге, кто честный, а кто нет.
Между ними можно делать очные ставки. Берется 2 чувака, и каждый из них говорит, честный ли второй. При этом честные чуваки всегда говорят правду, а нечестные могут говорить как правду, так и неправду.

Требуется за $\leq 2N$ очных ставок определить, кто честный, а кто нет (кто врот, а кто не врот).

\end{problem}
\begin{problem}

Есть массив из $2N + 1$ целых 32/64-битных чисел. $N$ чисел встречаются ровно по 2
раза, а одно встречается только 1 раз.

Требуется за 1 проход по массиву определить это число.

\end{problem}
\begin{problem}

Есть массив из $2N + 2$ целых 32/64-битных чисел. $N$ чисел встречаются ровно по 2
раза, а ДВА из них встречаются только 1 раз.

Требуется за $O(N)$ определить эти 2 числа.

\end{problem}
\begin{problem}

Есть массив из $N$ чисел. Известно, что есть число, встречающееся в нём больше, чем $\frac{N}{2}$ раз.

Требуется найти это число за 1 проход массива.

\end{problem}
\begin{problem}

Есть массив из $N$ чисел.

Придумать, как в нём найти все числа, встречающиеся больше, чем $\frac{N}{K}$ раз, за линейное относительно $N$ время.

Подсказка: сложность должна быть $O(N\log{K})$

\end{problem}
\begin{problem}

В одной стране есть деревня волшебников и деревня гномов.
Раз в год волшебники приходят в деревню к гномам и проводят ''зачистку''.

Для этого они выстраивают гномов в одну колонну по росту. При этом каждый гном видит только гномов меньшего роста, т.е. стоящих перед ним. 

На каждого гнома надевают либо белую, либо черную шляпу (предполагается, что и белых и черных шляп имеется бесконечно много).

Каждого гнома, начиная с самого высокого, спрашивают о том, какого цвета шляпа сейчас на нем.

Если гном ошибается, то его убивают (остальные гномы могут слышать ответ, но не знают, правильный он был или нет).

Какую стратегию нужно использовать гномам, чтобы минимизировать потери при ''зачистке''?
Сколько при оптимальной стратегии будет убито гномов?

\end{problem}
\begin{problem}

Есть круглый стол неизвестного диаметра и бесконечное число одинаковых круглых монет. Двое играют в игру по следущим правилам: на каждом своем ходу, игрок должен положить монету на стол так, чтобы она лежала на нем полностью и не перекрывала другие монеты на столе.

Выигрывает тот, кто делает последний возможный ход (после него больше нельзя положить монету на стол). 

Есть ли выигрышная стратегия для какого-нибудь из игроков?
Если да, то какая и кто при ней выигрывает?

\end{problem}
\begin{problem}

Есть строка длины $N$.

Требуется сделать циклический сдвиг строки на $M$ символов за линейное (относительно $N$) время, используя константное количество памяти.

\end{problem}
\begin{problem}

Есть строка длины $N$. В данной задаче будем считать словом любую последовательность символов, не содержащую пробела. В строке между словами может быть лишь один пробел.

Требуется инвертировать порядок слов в строке за линейное (относительно $N$) время, используя константное количество памяти.

\end{problem}
\begin{problem}

Есть прямоугольный торт с вырезанным из него прямоугольным куском произвольного размера и ориентации.

Как одним прямолинейным разрезом поделить торт на 2 равные по площади части?
\end{problem}
