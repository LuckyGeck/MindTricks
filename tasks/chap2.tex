\begin{problem}

В темной комнате стоит прямоугольный стол, в углах которого лежат монеты. Вам нужно зажечь свет. Свет загорится, если все монетки будут лежать вверх одинаковой стороной (либо орлом, либо решкой). 

За один ход можно перевернуть любое количество монеток, но при этом после каждого хода стол поворачивается произвольным образом. 

Так как же зажечь свет?

\end{problem}
\begin{problem}

Имеется 101 монета, из которых 50 фальшивых (монеты неотличимы по внешнему виду). Каждая фальшивая тяжелее настоящей на 1 грамм. Имеются также двухчашечные весы со стрелкой, показывающей разность весов на чашках. 

Вася Пупкин спрятал одну из монет. 

Может ли Петя Васечкин за одно взвешивание оставшихся монет определить, фальшива ли спрятанная монета?

\end{problem}
\begin{problem}

Найти сумму $\frac{1}{2\cdot3} + \frac{1}{3\cdot4}+\frac{1}{4\cdot5} + \ldots + \frac{1}{n\cdot (n-1)}$

\end{problem}
\begin{problem}

На хоккейном поле лежат три различные шайбы. Хоккеист бьёт по одной из них так, что она пролетает между двумя другими. Так он делает 25 раз. 

Могут ли после этого шайбы оказаться на исходных местах?

\end{problem}
\begin{problem}

В Стране Чудес на заколдованном озере семь островов, с каждого из них ведет один, три или пять мостов. 

Правда ли, что хотя бы один из мостов идет на берег? 

Может ли оказаться так, что на берег ведут ровно два моста?

\end{problem}
\begin{problem}

Докажите, что число людей, живших когда-либо на Земле и сделавших нечётное число рукопожатий - чётное.

\end{problem}
\begin{problem}

Некий лазутчик вознамерился проникнуть в стан неприятеля. Он искусно замаскировался в кустах и стал подслушивать какой пароль говорят охране лагеря. 

Вот кто-то подходит и часовой к нему обращаясь называет число: 

- Двадцать два. 

Несколько подумав посетитель отвечает: 

- Одиннадцать.

Часовой его пропускает.

Вот еще кто-то появляется и часовой к нему: 

- Двадцать шесть. 

Немного подумав, посетитель отвечает: 

- Тринадцать. 

Часовой его пропускает. 

«Ага» - осенило лазутчика, и он вылезает из кустов и уверенной походкой направляется к охране. 

– Сто, - говорит часовой. – Пятьдесят, отвечает лазутчик. 

И тут же попадает в цепкие объятья охраны: 

- Неправильно, три. Попался голубчик. 

Какой пароль был в этой части?  

\end{problem}
\begin{problem}

Есть 2007 монет, одна из которых фальшивая, отличающаяся от остальных по весу. 

Выясните, легче или тяжелее фальшивая монета, чем настоящая, при помощи двух взвешиваний.

\end{problem}
\begin{problem}

Умный продавец получил для продажи несколько пачек конвертов, по сто конвертов в пачке. 10 конвертов он отсчитывает за 10с. 

За сколько секунд он отсчитает 60 конвертов? 

А 180 конвертов?

\end{problem}